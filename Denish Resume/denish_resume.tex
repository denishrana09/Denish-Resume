%% start of file `template.tex'.
%% Copyright 2006-2013 Xavier Danaux (xdanaux@gmail.com).
%
% This work may be distributed and/or modified under the
% conditions of the LaTeX Project Public License version 1.3c,
% available at http://www.latex-project.org/lppl/.


\documentclass[11pt,a4paper,sans]{moderncv}        % possible options include font size ('10pt', '11pt' and '12pt'), paper size ('a4paper', 'letterpaper', 'a5paper', 'legalpaper', 'executivepaper' and 'landscape') and font family ('sans' and 'roman')

% modern themes
\moderncvstyle{banking}                            % style options are 'casual' (default), 'classic', 'oldstyle' and 'banking'
\moderncvcolor{blue}                                % color options 'blue' (default), 'orange', 'green', 'red', 'purple', 'grey' and 'black'
%\renewcommand{\familydefault}{\sfdefault}         % to set the default font; use '\sfdefault' for the default sans serif font, '\rmdefault' for the default roman one, or any tex font name
%\nopagenumbers{}                                  % uncomment to suppress automatic page numbering for CVs longer than one page

% character encoding
\usepackage[utf8]{inputenc}                       % if you are not using xelatex ou lualatex, replace by the encoding you are using
%\usepackage{CJKutf8}                              % if you need to use CJK to typeset your resume in Chinese, Japanese or Korean

% adjust the page margins
\usepackage[scale=0.82]{geometry}
%\setlength{\hintscolumnwidth}{3cm}                % if you want to change the width of the column with the dates
%\setlength{\makecvtitlenamewidth}{10cm}           % for the 'classic' style, if you want to force the width allocated to your name and avoid line breaks. be careful though, the length is normally calculated to avoid any overlap with your personal info; use this at your own typographical risks...

\usepackage{import}

% personal data
\name{Denish}{Rana}                      % optional, remove / comment the line if not wanted
\phone[mobile]{+91 9724510845}
\email{denishrana09@gmail.com}                               % optional, remove / comment the line if not wanted
                         % optional, remove / comment the line if not wanted
%\extrainfo{additional information}                 % optional, remove / comment the line if not wanted
%\photo[64pt][0.4pt]{picture}                       % optional, remove / comment the line if not wanted; '64pt' is the height the picture must be resized to, 0.4pt is the thickness of the frame around it (put it to 0pt for no frame) and 'picture' is the name of the picture file
%\quote{Some quote}                                 % optional, remove / comment the line if not wanted

% to show numerical labels in the bibliography (default is to show no labels); only useful if you make citations in your resume
%\makeatletter
%\renewcommand*{\bibliographyitemlabel}{\@biblabel{\arabic{enumiv}}}
%\makeatother
%\renewcommand*{\bibliographyitemlabel}{[\arabic{enumiv}]}% CONSIDER REPLACING THE ABOVE BY THIS

% bibliography with mutiple entries
%\usepackage{multibib}
%\newcites{book,misc}{{Books},{Others}}
%----------------------------------------------------------------------------------
%            content
%----------------------------------------------------------------------------------
\begin{document}
%\begin{CJK*}{UTF8}{gbsn}                          % to typeset your resume in Chinese using CJK
%-----       resume       ---------------------------------------------------------
\makecvtitle

\small{Currently in 4th year of B.Tech in Computer Engineering.I have good knowledge of Android Development, Web Development and Basic data structures and algorithms. }


%Personal Details
\section{Personal Details}
\vspace{6pt}
\begin{itemize}
\item \textbf{Date of Birth: }26 september 1997\vspace{4pt}
\item \textbf{Language known: }English, Hindi and Gujarati\vspace{4pt}.
\item \textbf{Permanent address: }C-18,Shiv Krupa Society,Ambanagar,Near U.M.Road,Surat,Gujarat,India-395002.\vspace{6pt}
\end{itemize}
%End Personal Details

\section{Education}
\vspace{6pt}
\begin{itemize}
\item{\cventry{August 2015 - July 2019(Expected)}{B.Tech in Computer Engineering, CPI: 6.99/10 (Semester 1-6)}{Dharmsinh Desai University, Nadiad}{}{}{\vspace{3pt}}}
\vspace{6pt}
\item{\cventry{June 2013 - April 2015}{Higher Secondary Education(HSC), Percentage: 74.46\%}{Jeevanbharti M.V.,Surat}{}{}{\vspace{3pt}}}
\vspace{6pt}
\item{\cventry{June 2012 - April 2013}{Secondary Education(SSC), Percentage: 85.83\%}{Jeevanbharti M.V.,Surat}{}{}{\vspace{3pt}}}
\end{itemize}

\section{Experience}
\vspace{6pt}
\begin{itemize}
\item \textbf{Zuru Tech,Ahmedabad} | \textit{Summer 2018} \vspace{3pt}\newline
\textit{Backend Engineer Intern} \newline
\small{Worked on project to help Interview process easy.} \newline
\small{Mostly worked with Nodejs, Worked with google cloud storage api, calendar api and git.} \vspace{6pt}
\end{itemize}

\section{Courses}
\vspace{6pt}
\begin{itemize}
\item Computer Programming\vspace{4pt}
\item Web Development\vspace{4pt}
\item Data Structures and Algorithms \vspace{4pt}
\item Database Management Systems\vspace{4pt}
\item Software Engineering\vspace{4pt}
\item Operating Systems\vspace{6pt}
\end{itemize}
 		

\section{Skills}
\vspace{6pt}
\begin{itemize}
\item \textbf{Proficient in:} JAVA, Android Development, Git, SQL, Nodejs, Data Structures, Algorithms\vspace{4pt}
\item \textbf{Basic ability with:} C, C\#, ASP.NET, MySQL, Object Oriented Programming, Oracle Database,  PHP, LaTeX, Assembly.

\vspace{6pt}
\end{itemize} 		


\section{Projects}
\vspace{6pt}


\begin{itemize}

\item \textbf{Blood Bank Android App} | \textit{February - March 2018} \vspace{3pt}\newline\textit{Using Java, Firebase, Google MapsAPI}\vspace{4pt}\newline \small{You have to register with your mobile number and blood group. whenever anyone wants help, he/she can search e.g.O+,AB+ and get the list of nearest persons in the result. you can make directly call the person from the result. App update users location every 1 km on real time database. App helps people in Emergency situations.} \href{https://github.com/denishrana09/BloodBankApp}{\textit{link}}\vspace{6pt}



\item{\textbf{College Festival Android App} | \textit{February 2018} \vspace{3pt}\newline\textit{Using JAVA}
\vspace{4pt}\newline
\small{Android Application made for College Festival named Felicific. It shows day wise technical,cultural and central-tech activities of festival. it has very good UI (Material Design) and has low storage on mobile.}}
\href{https://play.google.com/store/apps/details?id=in.opensol.felicific}{\textit{link}}\vspace{6pt}


\item \textbf{Online Travel Management System} | \textit{September 2017}\vspace{3pt}\newline\textit{Using C\#, ASP.NET}\vspace{4pt}\newline \small{Project is aimed to provide efficient and reliable Travel Management System. You can see how much you spent last month,last week. It has Great CSS.It uses hash function to store credentials.} 
\end{itemize}






\section{Co-curricular activity}

\vspace{6pt}

\begin{itemize}

\item{Part time competitive programmer on Codechef.}

\vspace{4pt}

\item{Made College Festival Android App }\vspace{6pt}

\item{Taken Part in SVNIT Hackathon }\vspace{6pt}

\end{itemize}


\section{Links}

\vspace{6pt}
 
\begin{itemize}

\item{\href{https://github.com/denishrana09}{GitHub:// \textbf{denishrana09}}}\vspace{4pt}
\item \href{https://stackoverflow.com/users/8018480/denish-rana}{Stackoverflow:// \textbf{denish-rana}}\vspace{4pt}
\item \href{https://www.linkedin.com/in/denish-rana/}{LinkedIn:// \textbf{denish-rana}}\vspace{6pt}
\item \href{https://www.interviewbit.com/profile/denishrana09}{InterviewBit:// \textbf{denishrana09}}\vspace{6pt}
\item \href{https://www.codechef.com/users/denishrana09}{Codechef:// \textbf{denishrana09}}\vspace{4pt}

\end{itemize}

% Publications from a BibTeX file without multibib
%  for numerical labels: \renewcommand{\bibliographyitemlabel}{\@biblabel{\arabic{enumiv}}}% CONSIDER MERGING WITH PREAMBLE PART
%  to redefine the heading string ("Publications"): \renewcommand{\refname}{Articles}
\nocite{*}
\bibliographystyle{plain}
\bibliography{publications}                        % 'publications' is the name of a BibTeX file


\end{document}


%% end of file `template.tex'.

